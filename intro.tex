% +++
% latex="texfot lualatex-dev"
% +++
\documentclass[body]{subfiles}
\begin{document}
\chapter{はじめに}
暴走温室状態は地球型惑星の多様な気候を理解するのに重要な概念である。
1 次元放射対流平衡モデルを用いた過去の研究により、海洋を持つ惑星の
大気上端から射出される外向き赤外放射 (OLR) には上限が存在すること
が知られている (Nakajima \etal 1992)。Nakajima \etal は、灰色大気を
もつ 1 次元放射対流平衡モデルで実験を行い、OLR に上限があることを
示した。そして Ishiwatari \etal 2002 ではそれを発展させて、非灰色
大気を持つ全球モデルを用いて実験を行い、そのモデルでも OLR に上限
があることを示した。灰色 3 次元モデルでは、太陽定数が増大して、熱の
供給の南北勾配が大きくなっても、潜熱によって熱が南北に輸送される効果
が大きくなり、OLR が南北に一様になるということも示した。

しかし一方で、非灰色の 3 次元モデルを用いた実験し、それについて考察
をしっかりと行った論文は無い状況であった。そこで、私の研究では、非灰色
の 3 次元全球モデルを用いて実験を行い、考察する。特に、太陽定数が
大きくなる時に、南北熱輸送の増加は何によって引き起こされるかを考察する。
水惑星が暴走温室状態になる直前には、Ishiwatari \etal 2002 が示した
ように、大気の状態が南北に均一化されている。大気の状態が南北に均一
になるためには、南北に熱が輸送されることが肝要である。であるから、
南北熱輸送が何によって引き起こされるかを明らかにすることで、暴走
温室状態に関する理解を深めることができると期待できる。
\end{document}
