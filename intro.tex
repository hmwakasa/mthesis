% +++
% latex="texfot lualatex-dev"
% +++
\documentclass[body]{subfiles}
\begin{document}
\chapter{はじめに}

\section{暴走温室状態}
暴走温室状態は地球型惑星の多様な気候を理解するのに重要な概念である。
暴走温室状態とは、惑星大気には射出できる放射に上限があって、
その上限より惑星に入射する放射の方が大きくなっている状態のことである。
この状態では、海が全て蒸発してしまうほど大気の温度が上昇すると考え
られている。1 次元放射対流平衡モデルを用いた過去の研究により、海洋
を持つ惑星の大気上端から射出される外向き赤外放射 (OLR) には上限が
存在することが知られている (Nakajima \etal*, 1992)。
OLR に上限値をもたらす大気構造の理解は、Nakajima \etal (1992) に
よって得られた。灰色大気をもつ 1 次元放射対流平衡モデルで
実験を行い、OLR には 2 種類の上限値が存在することを示した。
一つは、成層圏を通過できる放射の量から定まる限界であり、
もう一つは対流圏の射出構造から決まる値である。1 次元放射対流平衡
モデルで得られる OLR の上限値は、それらの限界値のうちの小さい方である。

Ishiwatari \etal (2002) では Nakajima \etal (1992) を発展させて、
灰色大気を持つ全球モデルを用いて実験を行った。
その結果、球面の効果と運動の効果を取り入れた 3 次元系でも OLR に上限
があることを示した。灰色 3 次元モデルでは、太陽定数が増大して、熱の
供給の南北勾配が大きくなっても、潜熱によって熱が南北に輸送される効果
が大きくなり、OLR が南北に一様になるということも示した。
太陽定数が増大すると南北に一様になる、という結果は、3 次元系の
暴走温室状態を考察する際、1 次元系で得られた結果を 3 次元に
適用できるということを示唆している。

\section{研究目的}
%しかし一方で、より地球に近い設定である、非灰色の 3 次元モデルを用いた
%実験し、それについて考察をしっかりと行った論文は無い状況であった。
この研究では、先行研究で行われた灰色 3 次元系のモデルを、より
地球に近い設定にするために、非灰色の 3 次元全球モデルを用いて考察する。
特に、太陽定数が
大きくなる時に、南北熱輸送の増加は何によって引き起こされるかを考察する。
水惑星が暴走温室状態になる直前には、Ishiwatari \etal 2002 が示した
ように、大気の状態が南北に均一化されている。大気の状態が南北に均一
になるためには、南北に熱が輸送されることが肝要である。であるから、
南北熱輸送が何によって引き起こされるかを明らかにすることで、暴走
温室状態に関する理解を深めることができると期待できる。
\end{document}
