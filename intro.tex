% +++
% latex="texfot lualatex-dev"
% +++
\documentclass[body]{subfiles}
\begin{document}
\chapter{はじめに}

\section{暴走温室状態}
暴走温室状態は地球型惑星の多様な気候を理解するのに重要な概念である。
暴走温室状態とは、惑星大気には射出できる放射に上限があって、
その上限より惑星に入射する放射の方が大きくなっている状態のことである。
この状態では、海が全て蒸発してしまうほど大気の温度が上昇すると考え
られている。1 次元放射対流平衡モデルを用いた過去の研究により、海洋
を持つ惑星の大気上端から射出される外向き赤外放射 (OLR) には上限が
存在することが知られている (Nakajima \etal*, 1992)。
OLR に上限値をもたらす大気構造の理解は、Nakajima \etal (1992) に
よって得られた。灰色大気をもつ 1 次元放射対流平衡モデルで
実験を行い、OLR には 2 種類の上限値が存在することを示した。
一つは、成層圏を通過できる放射の量から定まる限界であり、
もう一つは対流圏の射出構造から決まる値である。1 次元放射対流平衡
モデルで得られる OLR の上限値は、それらの限界値のうちの小さい方である。

Ishiwatari \etal (2002) では Nakajima \etal (1992) を発展させて、
灰色大気を持つ全球モデルを用いて実験を行った。
その結果、球面の効果と運動の効果を取り入れた 3 次元系でも OLR に上限
があり、その値がおよそ \(400\hmu{W/m^2}\)、太陽定数で言えば
\(1600\hmu{W/m^2}\) であることを示した。
灰色 3 次元モデルでは、太陽定数が増大して、熱の
供給の南北勾配が大きくなっても、潜熱によって熱が南北に輸送される効果
が大きくなり、OLR が南北に一様になるということも示した。
太陽定数が増大すると南北に一様になる、という結果は、3 次元系の
暴走温室状態を考察する際、1 次元系で得られた結果を 3 次元に
適用できるということを示唆している。すなわち、南北で一様化するので、
惑星大気に入射する平均フラックスが OLR の上限を超えると、暴走温室状態
になる。

しかし一方で、非灰色の 3 次元モデルを用いた実験を行ったものは、
石渡ら (2016) の学会発表程度しかなく、熱輸送の太陽定数依存性
に関してしっかりとした議論を行った論文は
無い状況であった。非灰色大気設定にするなど、放射スキームを変更した
場合、大気の温度が変わり、循環の強度が変わることが予想される。循環の
強度が変われば、熱の輸送も変わるので、Ishiwatari \etal (2002) で得られた
結果が、そのまま非灰色のモデルにも適用できるかは不明であった。

\section{研究目的}
そこで、この研究では、先行研究で行われた灰色 3 次元系のモデルを、非灰色の
3 次元全球モデルに変更して実験を行う。そして、太陽定数が大きくなる時に、
南北の熱輸送がどうなっているのかを調べる。非灰色大気の場合でも、Ishiwatari
\etal (2002) が示したように、太陽定数が大きくなったときに、南北熱輸送が
大きくなって、大気の状態が南北に均一化されていれば、非灰色大気であっても
OLR の上限を平均入射フラックスが超えていれば暴走温室状態になると言うこと
ができる。また、南北熱輸送が何によって引き起こされるかを明らかにすることで、
暴走温室状態に関する理解を深めることができると期待できる。

\section{本論文の構成}
本論文の構成を簡単に述べる。\ref{model} 章では、実験に用いたモデルと
実験の設定について述べる。\ref{result} 章では、実験で得られた結果を示し、
考察を行う。\ref{result} 章では、考察の結果をまとめ、今後の課題を示す。

\end{document}
