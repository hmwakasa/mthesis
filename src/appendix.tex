% +++
% latex="texfot lualatex-dev"
% +++
\documentclass[body]{subfiles}
\begin{document}
\chapter{南北熱輸送料を計算するスクリプトの解説}

この章では南北熱輸送を計算するスクリプトの解説を行う。

\section{潜熱輸送、乾燥静的エネルギー輸送、総熱輸送量を計算するスクリプト}

まず、図 \ref{EnFlx} を作成するスクリプトについて解説する。

このスクリプトでは、実験データから、東西風、南北風、気温、地表面気圧、比湿、
ジオポテンシャル高度を読み出し、それぞれの時刻ごとに以下の量を計算し、
\(f_L,f_D,f_T\) の値をひとつの NetCDF ファイルとして書き出す。
\begin{gather}
	f_D=\int^{2\pi}_0\int^1_0 (c_{pn}T+gz)\frac{p_s}{g}d\sigma\,a\cos\varphi\,d\varphi,\\
	f_L=\int^{2\pi}_0\int^1_0 Lq\frac{p_s}{g}d\sigma\,a\cos\varphi\,d\varphi,\\
	f_T=f_L+f_D.
\end{gather}
ここで、\(T,z\) は時刻 \(t\) の関数であるから、\(f_L,f_D,f_T\) も時刻の関数
になる。\(f_L,f_D,f_T\) を時間平均したものを描画すれば、図 \ref{EnFlx} と
同じ図が得られる。

\section{熱輸送の内訳を計算するスクリプト}

図 \ref{潜熱} や図 \ref{乾燥静的エネルギー}、図 \ref{潜熱移動性擾乱} や
図 \ref{潜熱平均子午面循環} などを作成するスクリプトについて解説する。

このスクリプトでは、実験データから 1 年分の東西風、南北風、気温、地表面気圧、
比湿、ジオポテンシャル高度のデータを読み出す。そして、全ての格子点で、全ての
時刻での \(c_{pn}T+gz, Lq, v\) の値を計算し、全ての座標において、それぞれの値の
時間平均と東西平均を計算し、それぞれの偏差も計算する。そして、NetCDF ファイルに
以下の値を格納する。
\begin{gather}
	[\bar v][\bar v]\\
	[\bar x^*\bar v^*]\\
	[\overline{x'v'}]
\end{gather}
ここで、\(x\) は \(c_{pn}T+gz\) または \(Lq\) である。それぞれの値は、\(\sigma\)
と \(\lambda\) の関数になっている。これを図にすれば図 \ref{潜熱平均子午面循環}
のような図が得られる。

また、別の NetCDF ファイルに、今計算した 6 つの値を \(\sigma\) で積分したものを
計算して書き出す。これらの値は \(\lambda\) のみの関数になっている。これを図にすれば
図 \ref{潜熱} のような図が得られる。

\end{document}
