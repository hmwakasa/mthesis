% +++
% latex="texfot lualatex-dev"
% +++
\documentclass[body]{subfiles}
\begin{document}
\begin{abstract}
	地球型惑星の多様な気候を理解する上で、暴走温室状態は重要な概念である。
	暴走温室状態とは、惑星の大気が射出する外向き赤外放射 (OLR) の上限より
	惑星に入射する放射の方が大きくなっている状態のことである。海洋を持つ
	惑星の OLR に上限が存在することは、Nakajima \etal (1992) が 1 次元
	放射対流平衡モデルを用いて示した。
	
	Nakajima \etal (1992) を発展させて、Ishiwatari \etal (2002) は 3 次元
	非灰色の全球モデルを用いて実験を行った。この研究では、3 次元非灰色の
	全球モデルでは、太陽定数を大きくすると、南北の熱輸送が大きくなって、
	大気の状態の南北差が減少し、太陽定数が \(1600\hmu{W/m^2}\) を超えると
	暴走温室状態になることを示した。大気の状態の南北差が減少して、一様化
	するので、平均入射フラックスが OLR 上限を超えると暴走するのである。
	
	Ishiwatari \etal (2002) が用いたモデルは、灰色でかつ雲がないという簡単な
	モデルであった。このモデルの大気を非灰色にするなど、放射スキームを変更
	した場合、大気の温度が変わり大気の循環構造が変わると考えられるため、
	Ishiwatari \etal (2002) で得られた結果が一般的なものなのかは不明であった。
	そこで、本研究では、3 次元非灰色モデル (DCPAM5) を用いて、雲あり・非灰色
	大気という設定で、太陽定数が増大したときの南北熱輸送がどのようになるのかを
	検討した。地球用放射スキーム
	(Chou and Lee, 1996; Chou \etal*, 1998) を用い、雲がある設定で
	5 通りの太陽定数 \(S=1366,1500,1600,1800,2000\hmu{W/m^2}\) で実験を行い、
	雲がない設定で 2 通りの太陽定数 \(S=1366,1500\hmu{W/m^2}\) で実験を行った。
	
	その結果、非灰色大気であっても OLR の分布や地表面温度の分布が太陽定数を
	増大させると南北差が小さくなるということがわかった。太陽定数が
	大きくなるにつれて、南北熱輸送が大きくなり、潜熱輸送も大きくなるが、
	太陽定数が \(1500\hmu{W/m^2}\) を超えると乾燥静的エネルギーの輸送が小さく
	なるということもわかった。つまり、太陽定数が大きくなると、非灰色大気でも
	乾燥静的エネルギーの南北輸送が減るものの、潜熱の南北輸送がそれにもまして
	大きくなるので、南北熱輸送が大きくなって南北差が減少するので、平均入射
	フラックスが OLRの上限を超えると暴走温室状態になると考えられる。
\end{abstract}
\end{document}
