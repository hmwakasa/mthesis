% +++
% latex="texfot lualatex-dev"
% +++
\documentclass[body]{subfiles}
\begin{document}
\chapter{モデルの概要}\label{model}

\section{系の設定と基礎方程式}
3 次元球殻上の3 次元球殻上の3 次元球殻上の大気大循環モデル DCPAM5 を用いて数値実験を行った。

DCPAM の力学過程で用いられている基礎方程式は以下の通りである。
\begin{gather}
	\pdv{\pi}{t}+\vb*{v}_H\cdot\nabla_\sigma\pi=-D-\pdv{\dot \sigma}{\sigma},\label{連続}\\
	\pdv{\Phi}{\sigma}=-\frac{RT_v}{\sigma},\label{静水圧}\\
	\pdv{\zeta}{t}=\frac{1}{a}\qty(\frac{1}{1-\mu^2}\pdv{V_A}{\lambda}-\pdv{U_A}{\mu})+\mathcal{D}[\zeta],\label{運動1}\\
	\pdv{D}{t}=\frac{1}{a}\qty(\frac{1}{1-\mu^2}\pdv{U_A}{\lambda})
		-\nabla^2_\sigma(\Phi+R\bar T\pi+KE)+\mathcal{D}[D],\label{運動2}\\
	\begin{split}
		\pdv{T}{t}=&-\frac{1}{a}\qty(\frac{1}{1-\mu^2}\pdv{UT'}{\lambda}+\pdv{VT'}{\mu})+T'D-\dot\sigma\pdv{T}{\sigma}\\
		&+\kappa T_v\qty(\pdv{\pi}{t}+\vb*{v}_H\cdot\nabla_\sigma\pi+\frac{\dot\sigma}{\sigma})
			+\frac{Q}{C_p}+\mathcal{D}[T]+\mathcal{D}'[\vb*{v}],
	\end{split}\label{熱力学}\\
	\pdv{q}{t}=-\frac{1}{a}\qty(\frac{1}{1-\mu^2}\pdv{U_q}{\lambda}+\pdv{V_q}{\mu})
		+qD-\dot\sigma\pdv{q}{\sigma}+S_q+\mathcal{D}[q].\label{水蒸気}
\end{gather}
ここで、鉛直座標には \(\sigma\) 座標系を用い、
それぞれ、連続の式 \eqref{連続}、静水圧の式 \eqref{静水圧}、
運動方程式 \eqref{運動1}, \eqref{運動2}、
熱力学の式 \eqref{熱力学}、水蒸気の式 \eqref{水蒸気} の式である。
各記号の意味は表 \ref{記号表} に記した。
\begin{table}[t]
	\centering
	\caption{記号表}\label{記号表}
	\begin{minipage}{\textwidth}
		\rule[0pt]{\textwidth}{\heavyrulewidth}\\
		\begin{minipage}{.5\textwidth}
			\hfill
			\begin{tabbing}
				\hspace*{3\zw}\=\kill
				\(\varphi,\lambda\)\>緯度経度\\
				\(\sigma\coloneqq p/p_s\)\quad\(\sigma\) 座標高度\\
				\(t\)\>時間\\
				\(\pi\coloneqq\ln[p_s]\)\\
				\(T\)\>気温\\
				\(q\)\>比湿\\
				\(a\)\>惑星半径\\
				\(f\)\>コリオリパラメータ\\
				\(\displaystyle\zeta\coloneqq\frac{1}{a}
					\qty(\frac{1}{1-\mu^2}\pdv{V}{\lambda}-\pdv{U}{\mu})\)\quad 渦度\\
				\(\displaystyle\zeta\coloneqq\frac{1}{a}
					\qty(\frac{1}{1-\mu^2}\pdv{U}{\lambda}+\pdv{V}{\mu})\)\quad 発散\\
				\(u,v\)\>東西・南北風速\\
				\((U,V)\coloneqq(u\cos\varphi,v\cos\varphi)\)\\
				\(\Phi\coloneqq gz\)\quad ジオポテンシャル高度\\
				\(\displaystyle\sigma\coloneqq\pdv{\sigma}{t}\equiv
					\pdv{\sigma}{t}+\frac{u}{a\cos\phi}\pdv{\sigma}{\lambda}
					+\frac{v}{a}\pdv{\sigma}{\phi}+\pdv{\sigma}{\sigma}\)\\
				\(\bar T\)\>基準温位\\
				\(T'\coloneqq T-\bar T\)\\
				\(T_v\coloneqq T(1+(\epsilon_v^{-1}-1)q)\)\\
				\(T'_v\coloneqq T_v-\bar T\)
			\end{tabbing}
		\end{minipage}
		\hfill
		\begin{minipage}{.5\textwidth}
			\hfill
			\begin{tabbing}
				\hspace*{3\zw}\=\kill
				\(\displaystyle U_A\coloneqq(\zeta+f)V-\dot\sigma\pdv{U}{\sigma}
					-\frac{RT'_v}{a}\pdv{\pi}{\lambda}+\mathcal{F}_\lambda\cos\phi\)\\
				\(\displaystyle V_A\coloneqq-(\zeta+f)U-\dot\sigma\pdv{V}{\sigma}
					\frac{RT'_v}{a}(1-\mu^2)\pdv{\pi}{\mu}+\mathcal{F}_\phi\cos\phi\)\\
				\(\displaystyle\vb*{v}_H\vdot\nabla_\sigma\pi\coloneqq
					\frac{U}{a(1-\mu^2)}\pdv{\theta}{\lambda}+\frac{V}{a}\pdv{\pi}{\lambda}\)\\
				\(\displaystyle\nabla_\sigma^2\coloneqq\frac{1}{a^2(1-\mu^2)}\pdv[2]{}{\lambda}
					+\frac{1}{a^2}\pdv{}{\mu}\qty[(1-\mu^2)\pdv{}{\mu}]\)\\
				\(\displaystyle KE\coloneqq\frac{U^2+V^2}{2(1-\mu^2)}\)\\
				\(\mathcal{D}[\zeta]\)\>渦度の水平発散とスポンジ層での散逸\\
				\(\mathcal{D}[D]\)\>発散の水平発散とスポンジ層での散逸\\
				\(\mathcal{D}[T]\)\>熱の水平拡散\\
				\(\mathcal{D}[q]\)\>水蒸気の水平拡散\\
				\(\mathcal{D}'[\vb*{v}]\)\>摩擦熱\\
				\(\mathcal{F}_\lambda\)\>小規模運動過程(経度方向)\\
				\(\mathcal{F}_\phi\)\>小規模運動過程(緯度方向)\\
				\(Q\)\>加熱・温度変化\\
				\(S_q\)\>水蒸気源\\
				\(R\)\>乾燥空気の気体定数\\
				\(c_{pn}\)\>乾燥空気の定圧比熱\\
				\(\kappa\coloneqq R/c_{pn}\)\\
				\(\epsilon_v\)\>水蒸気分子量比
			\end{tabbing}
		\end{minipage}\\
		\rule[0pt]{\textwidth}{\heavyrulewidth}
	\end{minipage}
\end{table}

放射過程には地球用放射モデルを用いている。考慮している大気成分は \ce{N2,CO2,H2O} である。
紫外・可視光・近赤外 (\(2600\)--\(57142.85\hmu{cm^{-1}}\))
は Chou and Lee (1996) に従って分割し、Toon \etal (1989) の手法を用いて放射伝達方程式を計算する。
\ce{H2O} の透過率は Chou and Lee (1996) による k 分布法のパラメータを利用して計算する。
雲の消散係数、単一散乱アルベド、非対称因子は Chou \etal (1998) の値を使用する。
レイリー散乱係数は Chou and Lee (1996) の値を使用する。
赤外 (\(0-3000\hmu{cm^{-1}}\)) は Chou \etal (2001) に従って 9 バンドに分割し、
散乱を無視した放射伝達方程式により計算する。\ce{H2O} の透過率は Chou \etal (2001)
の方法に基づき計算し、\ce{CO2} の低高度の透過率は Chou \etal (2001)、高高度の
透過率は Chou and Kouvaris (1991) の方法に基づいて、計算する。
雲の消散係数、単一散乱アルベド、非対称因子は Chou \etal (2001) の値を使用する。
サブグリッドスケールの混合・凝縮に関して、乱流混合は Mellor and Yamada level 2.5
(Mellor and Yamada, 1982) を使用する。また、Manabe \etal (1965) の乾燥対流調節スキームを用い、
積雲対流調節に関しては Relaxed Arakawa--Schubert (Moorthi and Suarez, 1992) を
使用する。
雲に関しては、移流・乱流混合・凝結による生成、時定数による消滅を考慮して雲水
混合比を予報する。惑星表面はスラブオーシャンであるとして、表面温度を計算する。

\section{実験設定}

であり、鉛直座標には \(\sigma\) 座標系を用い、その層数は 26 である。
実験で用いたモデルパラメータの値を、表 \ref{モデルパラメータ} に示す。
計算の水平分解能は、三角形切断の T42 に対応する、\(128\times64\)
であり、鉛直座標には \(\sigma\) 座標系を用い、その層数は 26 である。
ただし、太陽定数は表 \ref{実験リスト} に示す 5 通りで実験を行い、
\(S=1366,1500\hmu{W/m^2}\) では雲がある設定(雲時定数
\(13500\hmu{s}\))で 5 つの太陽定数で実験を行い、雲がない設定(雲時定数
\(0\hmu{s}\))で 2 つの太陽定数で実験を行った。

初期状態は、どの太陽定数においても、静止・等温 (\(280\hmu{K}\))・比湿は 0 で一様
とした。

\begin{table}[t]
	\centering
	\caption{モデルパラメータの値}\label{モデルパラメータ}
	\begin{tblr}{ll}
		\toprule
		モデルパラメータ&値\\
		\midrule
		惑星半径&\(a=6.37\hme{7}\hmu{m}\)\\
		自転角速度&\(\omega=7.292\hme{-5}\hmu{/s}\)\\
		重力加速度&\(g=9.8\hmu{m/s^2}\)\\
		乾燥空気の気体定数&\(R_n=287.1\hmu{J/kg/K}\)\\
		水蒸気の気体定数&\(R_v=461.5\hmu{J/kg/K}\)\\
		乾燥空気の定圧比熱&\(c_{pn}=1004\hmu{J/kg/K}\)\\
		水蒸気の定圧比熱&\(c_{pv}=1810\hmu{J/kg/K}\)\\
		乾燥空気の分子量&\(m_n=28.96\hme{-3}\hmu{kg/mol}\)\\
		水蒸気の分子量&\(m_v=18.02\hme{-3}\hmu{kg/mol}\)\\
		水の潜熱&\(L=2.50\hme{6}\hmu{J/kg}\)\\
		海のアルベド&\(A=0.1\)\\
		\bottomrule
	\end{tblr}
\end{table}


\begin{table}[t]
	\centering
	\caption{実験リスト}\label{実験リスト}
	\begin{tblr}{lccccc}
		\toprule
		実験名&太陽定数 \(S\ [\hmu*{W/m^2}]\)&雲時定数 \([\hmu*{s}]\)&
			積分期間 \([\text{年}]\)&計算結果を示す年度 \([\text{年度}]\)\\
		\midrule
		S1366&1366&13500&50&41\\
		S1500&1500&13500&20&11\\
		S1600&1600&13500&20&11\\
		S1800&1800&13500&20&11\\
		S2000&2000&13500&30&21\\
		S1366nc&1366&0&11&11\\
		S1500nc&1500&0&11&11\\
		\bottomrule
	\end{tblr}
\end{table}

\end{document}

