% +++
% latex="texfot lualatex-dev"
% +++
\documentclass[body]{subfiles}
\begin{document}
\chapter{モデルの概要}

\section{系の設定と基礎方程式}
3 次元球殻上の3 次元球殻上の3 次元球殻上の大気大循環モデル DCPAM5 を用いて数値実験を行った。

連続の式 \eqref{連続}、静水圧の式 \eqref{静水圧}、運動方程式 \eqref{運動1}, \eqref{運動2}、
熱力学の式 \eqref{熱力学}、水蒸気 \eqref{水蒸気} の式は以下の通りである。各記号の意味は表
\ref{記号表} に記した。
\begin{gather}
	\pdv{\pi}{t}+\vb*{v}_H\cdot\nabla_\sigma\pi=-D-\pdv{\dot \sigma}{\sigma},\label{連続}\\
	\pdv{\Phi}{\sigma}=-\frac{RT_v}{\sigma},\label{静水圧}\\
	\pdv{\zeta}{t}=\frac{1}{a}\qty(\frac{1}{1-\mu^2}\pdv{V_A}{\lambda}-\pdv{U_A}{\mu})+\mathcal{D}[\zeta],\label{運動1}\\
	\pdv{D}{t}=\frac{1}{a}\qty(\frac{1}{1-\mu^2}\pdv{U_A}{\lambda})
		-\nabla^2_\sigma(\Phi+R\bar T\pi+KE)+\mathcal{D}[D],\label{運動2}\\
	\begin{split}
		\pdv{T}{t}=&-\frac{1}{a}\qty(\frac{1}{1-\mu^2}\pdv{UT'}{\lambda}+\pdv{VT'}{\mu})+T'D-\dot\sigma\pdv{T}{\sigma}\\
		&+\kappa T_v\qty(\pdv{\pi}{t}+\vb*{v}_H\cdot\nabla_\sigma\pi+\frac{\dot\sigma}{\sigma})
			+\frac{Q}{C_p}+\mathcal{D}[T]+\mathcal{D}'[\vb*{v}],
	\end{split}\label{熱力学}\\
	\pdv{q}{t}=-\frac{1}{a}\qty(\frac{1}{1-\mu^2}\pdv{U_q}{\lambda}+\pdv{V_q}{\mu})
		+qD-\dot\sigma\pdv{q}{\sigma}+S_q+\mathcal{D}[q].\label{水蒸気}
\end{gather}
\begin{table}[t]
	\centering
	\caption{記号表}\label{記号表}
	\begin{minipage}{.8\textwidth}
		\rule[0pt]{\textwidth}{\heavyrulewidth}\\
		\begin{minipage}{.5\textwidth}
			\hfill
			\begin{tabbing}
				\hspace*{3\zw}\=\kill
				\(\varphi,\lambda\)\>緯度経度\\
				\(\sigma:=p/p_s\)\quad\(\sigma\) 座標高度\\
				\(t\)\>時間\\
				\(\pi:=\ln[p_s]\)\\
				\(T\)\>気温\\
				\(q\)\>比湿\\
				\(a\)\>惑星半径\\
			\end{tabbing}
		\end{minipage}
		\hfill
		\begin{minipage}{.5\textwidth}
			\hfill
			\begin{tabbing}
				\hspace*{3\zw}\=\kill
				\(\displaystyle\zeta:=\frac{1}{a}\qty(\frac{1}{1-\mu^2}\pdv{V}{\lambda}-\pdv{U}{\mu})\)\quad 渦度\\
				\(\displaystyle\zeta:=\frac{1}{a}\qty(\frac{1}{1-\mu^2}\pdv{U}{\lambda}+\pdv{V}{\mu})\)\quad 発散\\
				\(u,v\)\>東西・南北風速\\
				\((U,V):=(u\cos\varphi,v\cos\varphi)\)\\
				\(\mathcal{D}\)\>水平拡散\\
				\(\mathcal{D}'[\vb*{v}]\)\>摩擦熱
			\end{tabbing}
		\end{minipage}\\
		\rule[0pt]{\textwidth}{\heavyrulewidth}
	\end{minipage}
\end{table}

放射過程には地球用放射モデルを用いている。紫外・可視光・近赤外 (\(2600\)--\(57142.85\hmu{cm^{-1}}\))
は \(1000\)--\(57142.85\hmu{cm^{-1}}\) を 11 バンドに分割 (Chou and Lee, 1996) し、
\(\delta\)-Eddington 近似した放射伝達方程式により計算をする (Toon \etal*, 1989)。
\ce{H2O} の透過率は Chou and Lee (1996) による k 分布法のパラメータを利用して計算する。
雲の消散係数、単一散乱アルベド、非対称因子は Chou \etal (1998) の値を使用する。
レイリー散乱係数と \ce{O3} の吸収係数は Chou and Lee (1996) の値を使用する。
赤外 (\(0-3000\hmu{cm^{-1}}\)) は Chou \etal (2001) に従って 9 バンドに分割し、
散乱を無視した放射伝達方程式により計算する。\ce{H2O,CH4,N2O} の透過率は Chou \etal (2001)
の方法に基づき計算し、\ce{CO2} の低高度の透過率は Chou \etal (2001)、高高度の
透過率は Chou and Kouvaris (1991) の方法に基づいて、\ce{O3} の透過率は
Chou and Kouvaris (1991) の方法に基づいて計算する。
雲の消散係数、単一散乱アルベド、非対称因子は Chou \etal (2001) の値を使用する。

サブグリッドスケールの混合・凝縮に関して、乱流混合は \textbf{???} を使用する。
また、Manabe \etal (1965) の乾燥対流調節スキームを用い、
積雲対流調節に関しては Relaxed Arakawa--Schubert (Moorthi and Suarez, 1992) を
使用する。

雲に関しては、移流・乱流混合・凝結による生成、定数時定数による消滅を考慮して雲水
混合比を予報する。

\section{実験設定}

\begin{table}[t]
	\centering
	\caption{モデルパラメータの値}\label{モデルパラメータ}
	\begin{tblr}{ll}
		\toprule
		モデルパラメータ&値\\
		\midrule
		惑星半径&\(a=6.37\hme{7}\hmu{m}\)\\
		自転角速度&\(\omega=7.292\hme{-5}\hmu{/s}\)\\
		重力加速度&\(g=9.8\hmu{m/s^2}\)\\
		乾燥空気の気体定数&\(R_n=287.1\hmu{J/kg/K}\)\\
		水蒸気の気体定数&\(R_v=461.5\hmu{J/kg/K}\)\\
		乾燥空気の定圧比熱&\(c_{pn}=1004\hmu{J/kg/K}\)\\
		水蒸気の定圧比熱&\(c_{pv}=1810\hmu{J/kg/K}\)\\
		乾燥空気の分子量&\(m_n=28.96\hme{-3}\hmu{kg/mol}\)\\
		水蒸気の分子量&\(m_v=18.02\hme{-3}\hmu{kg/mol}\)\\
		水の潜熱&\(L=2.50\hme{6}\hmu{J/kg}\)\\
		海のアルベド&\(A=0.1\)\\
		\bottomrule
	\end{tblr}
\end{table}

\begin{table}[t]
	\centering
	\caption{実験を行った太陽定数の値}\label{太陽定数}
	\begin{tblr}{lccccc}
		\toprule
		太陽定数 \(S\ [\hmu*{W/m^2}]\)&1300&1500&1600&1800&2000\\
		積分期間 \([\text{年}]\)&50&20&20&20&30\\
		計算結果を示す年度 \([\text{年度}]\)&41&11&11&11&21\\
		\bottomrule
	\end{tblr}
\end{table}

実験で用いたモデルパラメータの値を、表 \ref{モデルパラメータ} に示す。
本研究で行う計算の水平分解能は、三角形切断の T42 に対応する、\(128\times64\)
であり、鉛直座標には \(\sigma\) 座標系を用い、その層数は 26 である。

表 \ref{太陽定数} に示す太陽定数で実験を行った。以後、各実験で与えた太陽定数
の値に、S を前置したものを実験名とする。

初期状態は、どの太陽定数においても、静止・等温 (\(280\hmu{K}\))・比湿は 0 で一様
とした。

\end{document}

