% +++
% latex="texfot lualatex-dev"
% +++
\documentclass[body]{subfiles}
\begin{document}
\chapter{結論と今後の展望}

非灰色 3 次元モデルを用いて数値実験を行い、太陽定数の増大に伴って
南北熱輸送がどのようになるのかを調べた。

今回の研究により、非灰色の 3 次元モデルを用いた場合でも、
大気の構造が南北に一様化するという、Ishiwatari \etal (2002) と
同様な結論を得た。

また、太陽定数が大きくなるにつれて、南北総熱輸送量は大きくなるが、
乾燥静的エネルギー輸送が \(S=1500\hmu{W/m^2}\) 以上では小さくなる
こともわかった。一方で、潜熱輸送は太陽定数が大きくなるにつれて
大きくなるので、太陽定数が大きい場合は潜熱輸送の影響が大きいという
事もわかった。さらに、太陽定数が大きいときの潜熱輸送の大部分は、
移動性の擾乱によってもたらされることもわかった。

この研究で得られた、乾燥静的エネルギーの輸送量が \(S=1500\hmu{W/m^2}\)
でピークを持つということは、先行研究では得られていない結果である。
この研究結果は、雲があるモデルでの計算結果であるため、雲がないモデルで
実験を行った Ishiwatari \etal (2002) と単純には比較ができない。
そこで、雲の時定数を 0 とした実験もする必要があるだろう。




\end{document}
