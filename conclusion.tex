% +++
% latex="texfot lualatex-dev"
% +++
\documentclass[body]{subfiles}
\begin{document}
\chapter{結論と今後の展望}\label{conclusion}

非灰色 3 次元モデルを用いて数値実験を行い、太陽定数の増大に伴って南北熱輸送
がどのようになるのかを調べた。

今回の研究により、非灰色の 3 次元モデルを用いた場合でも、大気の構造が南北に
一様化するという、Ishiwatari \etal (2002) と同様な結論を得た。
また、太陽定数が大きくなるにつれて、南北総熱輸送量は大きくなるが、乾燥静的
エネルギー輸送が、雲がある場合は \(S=1600\hmu{W/m^2}\) 以上では小さくなる
こともわかった。一方で、潜熱輸送は太陽定数が大きくなるにつれて大きくなるので、
太陽定数が大きい場合は潜熱輸送の影響が大きいという事もわかった。さらに、
太陽定数が大きいときの潜熱輸送の大部分は、移動性の擾乱によってもたらされること
もわかった。

雲の時定数を 0 とした実験も行った。雲がない場合は、雲がある場合と
比較して南北熱輸送が大きくなることがわかった。また、雲がある場合と同様に、
太陽定数が大きくなると乾燥静的エネルギーの輸送が小さくなるが、それ以上に
潜熱の輸送が大きくなることもわかった。そして、太陽定数が大きいときの潜熱輸送
は、雲がない場合でも雲がある場合と同じく、移動性の擾乱によってもたらされる
ことがわかった。

以上のことから、非灰色の大気であっても、太陽定数が大きくなると、乾燥静的
エネルギーの南北輸送が小さくなるものの、それにもまして潜熱の輸送が大きく
なることで南北差が小さくなる。そのため、3 次元非灰色のモデルでも、平均
入射フラックスが OLR の上限値を超えたときに、暴走温室状態になっていると定義
することができる。

しかし、雲がない場合では \(S=1500\hmu{W/m^2}\)
より大きい太陽定数では計算を行っておらず、\(S=1500\hmu{W/m^2}\) の場合でも
平衡に達するのに十分な時間の積分を行っていない。したがって、雲がない設定で、
より大きい太陽定数を用いて計算をしたり、積分期間を長くしたりしてさらに検討
を重ねなければならないだろう。
また、実験 S1800, S2000, S1500nc では、ジェットが大気上端に達しており、正しく
大気の状態が表現されていない可能性があった。したがって、太陽定数が大きい
場合では、鉛直層数を増やした計算をして、更に検討をする必要があるだろう。

\end{document}
