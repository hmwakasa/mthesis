% +++
% latex="texfot lualatex-dev"
% +++
\documentclass[body]{subfiles}
\begin{document}
\chapter{結論と今後の展望}\label{conclusion}

非灰色 3 次元モデルを用いて数値実験を行い、太陽定数の増大に伴って南北熱輸送
がどのようになるのかを調べた。

今回の研究により、非灰色の 3 次元モデルを用いた場合でも、大気の構造が南北に
一様化するという、Ishiwatari \etal (2002) と同様な結論を得た。

また、太陽定数が大きくなるにつれて、南北総熱輸送量は大きくなるが、乾燥静的
エネルギー輸送が、雲がある場合は \(S=1500\hmu{W/m^2}\) 以上では小さくなる
こともわかった。一方で、潜熱輸送は太陽定数が大きくなるにつれて大きくなるので、
太陽定数が大きい場合は潜熱輸送の影響が大きいという事もわかった。さらに、
太陽定数が大きいときの潜熱輸送の大部分は、移動性の擾乱によってもたらされること
もわかった。

この研究で得られた、雲がある場合は乾燥静的エネルギーの輸送量が \(S=1500\hmu{W/m^2}\)
でピークを持つということは、先行研究では得られていない結果である。この結果は、
雲がないモデルで実験を行った Ishiwatari \etal (2002) と単純には比較ができない。
そこで、雲の時定数を 0 とした実験も行ったが、雲がない場合では \(S=1500\hmu{W/m^2}\)
より大きい太陽定数では計算を行っておらず、\(S=1500\hmu{W/m^2}\) の場合でも
平衡に達するのに十分な時間の積分を行っていない。したがって、雲がない設定で、
より大きい太陽定数を用いて計算をしたり、積分期間を長くしたりしてさらに検討
を重ねなければならないだろう。

また、実験 S1800, S2000, S1500nc では、ジェットが大気上端に達しており、正しく
大気の状態が表現されていない可能性があった。したがって、太陽定数が大きい
場合では、鉛直層数を増やした計算をして、更に検討をする必要があるだろう。

\end{document}
