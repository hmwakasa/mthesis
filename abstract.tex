% +++
% latex="texfot lualatex-dev"
% +++
\documentclass[body]{subfiles}
\begin{document}
\begin{abstract}
	地球型惑星の多様な気候を理解する上で、暴走温室状態は重要な概念である。
	Nakajima \etal (1992) は、1 次元放射対流平衡モデルを用いて海洋を持つ
	惑星の大気が射出できる外向き赤外放射 (OLR) には上限があると示した。
	暴走温室状態とは、OLR の上限より惑星に入射する放射の方が大きく
	なっている状態のことである。この状態では、海が全て蒸発してしまうほど
	大気の温度が上昇すると考えられている。
	
	Nakajima \etal (1992) を発展させて、3 次元非灰色の全球モデルでも OLR に
	上限が現れるということを Ishiwatari \etal (2002) が示した。この研究では、
	3 次元灰色モデルでも OLR に上限が現れることを示したのと同時に、太陽定数
	が増大して熱の供給の南北差が大きくなったとしても、熱の輸送が大きくなって
	大気の状態が南北に一様化してゆくということも示した。
	
	Ishiwatari \etal (2002) が用いたモデルは、灰色でかつ雲がないという簡単な
	モデルであった。そこで、本研究では、3 次元非灰色モデル (DCPAM5) を用いて、
	雲あり・非灰色大気というより地球に近い設定で、太陽定数が増大したときの
	南北熱輸送がどのようになるのかを検討した。地球用放射スキーム
	(Chou and Lee, 1996; Chou \etal*, 1998) を用い、雲がある設定で
	5 通りの太陽定数 \(S=1366,1500,1600,1800,2000\hmu{W/m^2}\) で実験を行い、
	雲がない設定で 2 通りの太陽定数 \(S=1366,1500\hmu{W/m^2}\) で実験を行った。
	その結果、非灰色大気であっても OLR の布や地表面温度の分布が太陽定数を
	増大させると南北に一様化してゆくということがわかった。また、太陽定数が
	大きくなるにつれて、南北熱輸送が大きくなり、潜熱輸送も大きくなるが、
	太陽定数が \(1500\hmu{W/m^2}\) を超えると乾燥静的エネルギーの輸送が小さく
	なるということもわかった。つまり、太陽定数が大きくなると潜熱輸送の影響が
	大きくなるのである。しかし、乾燥静的エネルギーの輸送が \(1500\hmu{W/m^2}\)
	でピークを持つなど、潜熱輸送以外にも、複雑な過程が含まれていると考えられる。
\end{abstract}
\end{document}
