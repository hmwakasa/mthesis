% +++
% latex="texfot lualatex-dev"
% +++
\documentclass[body]{subfiles}
\begin{document}
\begin{abstract}
	地球型惑星の多様な気候を理解する上で、暴走温室状態は重要な概念である。
	暴走温室状態とは、惑星大気が射出する放射より、惑星に入射する放射の方が大きく
	なっている状態のことである。暴走温室状態に関して、Nakajima \etal (1992) が
	1 次元放射対流平衡モデルを用いて海洋を持つ惑星の大気が射出できる外向き赤外
	放射 (OLR) には上限があると示した。
	
	それを発展させて、3 次元非灰色の全球モデルでも OLR に上限が現れるという
	ことを Ishiwatari \etal (2002) が示した。この研究では、3 次元灰色モデル
	でも OLR に上限が現れることを示したのと同時に、太陽定数が増大して熱の供給の
	南北差が大きくなったとしても、熱の輸送が大きくなって大気の状態が南北に
	一様化してゆくということも示した。
	
	Ishiwatari \etal (2002) が用いたモデルは、灰色でかつ雲がないという簡単な
	モデルであった。そこで、本研究では、3 次元非灰色モデル (DCPAM5) を用いて、
	雲あり・非灰色大気というより地球に近い設定で、太陽定数が増大したときの
	南北熱輸送がどのようになるのかを検討した。その結果、非灰色大気であっても、
	OLR の分布や地表面温度の分布が、太陽定数を増大させると南北に一様化してゆく
	ということがわかった。また、太陽定数が大きくなるにつれて、南北熱輸送が
	大きくなり、潜熱輸送も大きくなるが、太陽定数が \(1500\hmu{W/m^2}\) を
	超えると乾燥静的エネルギーの輸送が小さくなるということもわかった。
	つまり、太陽定数が大きくなると潜熱輸送の影響が大きくなるのである。
\end{abstract}
\end{document}
